\documentclass{article}
%\usepackage[T1]{fontenc}
%\usepackage{amssymb, amsmath, graphicx, subfigure, enumerate}
%\usepackage{amsthm,alltt} 
\usepackage[margin=1.25in]{geometry} %geometry (sets margin) and other useful packages
\usepackage{graphicx,ctable,booktabs}
\usepackage{mathtools}
\usepackage[boxed]{algorithm2e}
\usepackage{mathdots}
\usepackage{fancyhdr} %Fancy-header package to modify header/page numbering
\usepackage{cleveref}

\setlength{\oddsidemargin}{.25in}
\setlength{\evensidemargin}{.25in}
\setlength{\textwidth}{6in}
\setlength{\topmargin}{-0.4in}
\setlength{\textheight}{8.5in}



\newcommand{\heading}[6]{
  \renewcommand{\thepage}{\arabic{page}} % used to be #1-\arabic{page}
  \noindent
  \begin{center}
  \framebox{
    \vbox{
      \hbox to 5.78in { \textbf{#2} \hfill #3 }
      \vspace{4mm}
      \hbox to 5.78in { {\Large \hfill #6  \hfill} }
      \vspace{2mm}
      \hbox to 5.78in { \textit{{Name: }} }
    }
  }
  \end{center}
  \vspace*{4mm}
}

%Redefining sections as problems
\makeatletter

\newenvironment{problem}{\@startsection
       {section}
       {2}
       {-.2em}
       {-3.5ex plus -1ex minus -.2ex}
       {2.3ex plus .2ex}
       {\pagebreak[3]%forces pagebreak when space is small; use \eject for better results
       \large\bf\noindent{Problem }
       }
       }
       %{%\vspace{1ex}\begin{center} \rule{0.3\linewidth}{.3pt}\end{center}}
       %\begin{center}\large\bf \ldots\ldots\ldots\end{center}}
\makeatother


\newtheorem{theorem}{Theorem}[section]
\newtheorem{definition}[theorem]{Definition}
\newtheorem{remark}[theorem]{Remark}
\newtheorem{lemma}[theorem]{Lemma}
\newtheorem{corollary}[theorem]{Corollary}
\newtheorem{proposition}[theorem]{Proposition}
\newtheorem{claim}[theorem]{Claim}
\newtheorem{observation}[theorem]{Observation}
\newtheorem{fact}[theorem]{Fact}
\newtheorem{assumption}[theorem]{Assumption}

%\newenvironment{proof}{\noindent{\bf Proof:} \hspace*{1mm}}{
% \hspace*{\fill} $\Box$ }
%\newenvironment{proof_of}[1]{\noindent {\bf Proof of #1:}
% \hspace*{1mm}}{\hspace*{\fill} $\Box$ }
%\newenvironment{proof_claim}{\begin{quotation} \noindent}{
% \hspace*{\fill} $\diamond$ \end{quotation}}

\newcommand{\problemset}[3]{\heading{#1}{\classname}{#2}{\studentname}{#3}{Problem Set #1}} % Don't change this line
%%%%%%%%%%%%%%%%%%%%%%%%%% Change this stuff below, don't change the line above this one
\newcommand{\problemsetnum}{1}            % problem set number
\newcommand{\duedate}{Due: Jan. 15, 2018, 8am EST} % problem set deadline
\newcommand{\studentname}{Student Name: Michael Becker}      % name of student (i.e., you)
\newcommand{\classname}{Name:  Michael Becker }
%\newcommand{\instructor}{Prof. Eric Vigoda}
%%%%%%%%%%%%%%%%%%%%%%%%%%

\pagestyle{fancy}
%\addtolength{\headwidth}{\marginparsep} %these change header-rule width
%\addtolength{\headwidth}{\marginparwidth}
\lhead{\classname} %Problem \thesection}
\chead{} 
\rhead{\thepage} 
%\lfoot{\small\scshape \classname}
%\cfoot{} 
%\rfoot{\footnotesize PS \#\problemsetnum} 
\renewcommand{\headrulewidth}{.3pt} 
\renewcommand{\footrulewidth}{.3pt}
\setlength\voffset{-0.25in}
\setlength\textheight{648pt}


\newcommand{\sit}{\shortintertext}
\newcommand\deq{\mathrel{\overset{\makebox[0pt]{\mbox{\normalfont\tiny\sffamily def}}}{=}}}
\newcommand{\ones}{\mathbbm{1}}
\newcommand{\e}{\vec{e}}
\newcommand{\tr}{\text{tr}}
\newcommand{\bs}{\boldsymbol}
\mathchardef\mhyphen="2D
\newcommand{\C}{\mathbb{C}}
\newcommand{\R}{\mathbb{R}}
\newcommand{\II}{\mathcal{I}}
\newcommand{\FF}{\mathcal{F}}
\newcommand{\X}{\mathcal{X}}
\newcommand{\Y}{\mathcal{Y}}
\newcommand{\ra}{\rightarrow}
\newcommand{\Ra}{\Rightarrow}
\newcommand{\PP}{\mathbb{P}}
\newcommand{\sse}{\subseteq}
\newcommand{\eps}{\epsilon}
\newcommand{\N}{\mathcal{N}}
\newcommand{\poly}{\textup{poly}}

\newcommand{\dom}{\textup{dom}}

\renewcommand{\thesubsection}{\thesection.\roman{subsection}}


% auto sized delimiters
\newcommand{\Br}[1]{\left\{#1\right\}}
\newcommand{\br}[1]{\left[#1\right]}
\newcommand{\pr}[1]{\left(#1\right)}
\newcommand{\ceil}[1]{\left\lceil#1\right\rceil}
\newcommand{\floor}[1]{\left\lfloor#1\right\rfloor}
\newcommand{\abs}[1]{\left|#1\right|}
\newcommand{\sgn}{\textup{sgn}}

%default delimiter for Pr and E
\DeclarePairedDelimiter{\defaultDelim}{[}{]}

\DeclareMathOperator{\capPr}{Pr}
\renewcommand{\Pr}[2][]{\capPr_{#1}\defaultDelim*{#2}}
\DeclareMathOperator{\capE}{E}
\newcommand{\E}[2][]{\capE_{#1}\defaultDelim*{#2}}
\DeclareMathOperator{\capVar}{Var}
\newcommand{\Var}[2][]{\capVar_{#1}\defaultDelim*{#2}}

\newcommand{\vs}{\vspace{.1in}}
\newcommand{\vB}{\vspace{.3in}}

%\DeclareMathOperator*{}{} puts subscript below


%%%%%%%%%%%%%%%%%%%%%%%%%%%%%%%%%%%%%%%%%%%%%%%%%
\begin{document}
{\bf \noindent Homework 1. Part 2 \\ Due: Monday, January 15, 2018 before 8am EST.}

\begin{problem}{[DPV] 6.2 -- Hotel stops with minimum penalty.}


{\bf (a) } Define the entries of your table in words.  E.g., $T(i)$ or $T(i,j)$ is~....

\vspace{.25in}
Let p(i) be the minimum penalty on $a_1, a_2, ... , a_i$ where $a_i$ is included. 

\vspace{3in}

{\bf (b) } State recurrence for entries of table in
terms of smaller subproblems.

\vspace{.25in}
$p(i) = \min_j \{p(j) + (200 - (a_i-a_j))^2\}$ where $j <  i$

\newpage

{\bf (c) }  Write pseudocode for your algorithm to solve this problem.

\begin{algorithm}
	\SetKwInOut{Input}{Input}
    	\SetKwInOut{Output}{Output}
   	\Input{an array $a$ of size $N$}
	\Output{the minimum penalty by taking the optimal path of hotels}
	\BlankLine
	
	initialize array $p$ of size $N$ to 0\;
	\For{$i\in 1 \longrightarrow N$}{
		$min\_j = i-1$ \;	
		\For{$j\in 0 \longrightarrow i-2$}{
			\If{$p[j] + (200 - (a[i] - a[j]))^2 < p[min\_j] + (200 - (a[i] - a[min\_j]))^2$}{
				$min\_j = j$
			}
		}
		$p[i] = p[min\_j] + (200 - (a[i] - a[min\_j]))^2$
	}
	\KwRet{last element in array $p$}
\end{algorithm}

\vspace{2in}

{\bf (d) } Analyze the running time of your algorithm.

\vspace{.25in}
Since there are two for loops (one looping through N items and the other going back to find the minimum value), it will run in $O(n^2)$
 

\end{problem}

\newpage

\begin{problem}{[DPV] 6.3 -- Yuckdonald's}


{\bf (a) } Define the entries of your table in words.  E.g., $T(i)$ or $T(i,j)$ is~....

\vspace{.25in}
Let z(i) be the maximum profit in $\left\{\begin{matrix}
m_1, m_2, ... , m_i\\ 
p_1, p_2, ... , p_i
\end{matrix}\right.$ where $m_i$ and $p_i$ are included. 

\vspace{3in}

{\bf (b) } State recurrence for entries of table in
terms of smaller subproblems.

\vspace{.25in}
$z(i) = p_i + \max_j\{ z_j : (m_i - m_j) < k \}$

\newpage

{\bf (c) }  Write pseudocode for your algorithm to solve this problem.

\begin{algorithm}
	\SetKwInOut{Input}{Input}
    	\SetKwInOut{Output}{Output}
   	\Input{an array $m$ and array $p$ of size $N$ and minimum distance $k$}
	\Output{the maximum profit gathered from optimal store placement}
	\BlankLine
	
	initialize array $z$ of size $N$ to 0\;
	\For{$i\in 1 \longrightarrow N$}{
		$max\_j = 0$ \;	
		\For{$j\in 0 \longrightarrow i$}{
			\If{$p[j] > p[min\_j]$}{
				$max\_j = j$
			}
		}
		\eIf{$max\_j$ != None and $(m[i] - m[j]) >= k$}{
			z[i] = p[i] + p[max\_j]
		}{
			z[i] = p[i]
		}
	}
	
	$max\_profit = 0$
	\For{$i\in 1 \longrightarrow N$}{
		\If{$z[i] > z[max\_profit]$}{
			$max\_profit = i$
		}
	}
	\KwRet{return z[max\_profit]}
\end{algorithm}

\vspace{2in}

{\bf (d) } Analyze the running time of your algorithm.

\vspace{.25in}
Since there are two for loops (one looping through N items and the other going back to find the minimum value), it will run in $O(n^2)$
 

\end{problem}


\newpage

\begin{problem}{[DPV] 6.17 -- Coin changing (unlimited supply of each denomination) Practice Only}


{\bf (a) } Define the entries of your table in words.  E.g., $T(i)$ or $T(i,j)$ is~....

\vspace{3in}

{\bf (b) } State recurrence for entries of table in
terms of smaller subproblems.


\newpage

{\bf (c) }  Write pseudocode for your algorithm to solve this problem.

\vspace{4in}

{\bf (d) } Analyze the running time of your algorithm.

\end{problem}
 
\newpage

\begin{problem}{[DPV] 6.18 -- Coin changing (use each denomination at most once) Practice Only}


{\bf (a) } Define the entries of your table in words.  E.g., $T(i)$ or $T(i,j)$ is~....

\vspace{3in}

{\bf (b) } State recurrence for entries of table in
terms of smaller subproblems.


\newpage

{\bf (c) }  Write pseudocode for your algorithm to solve this problem.

\vspace{4in}

{\bf (d) } Analyze the running time of your algorithm.


 

\end{problem}


\end{document}